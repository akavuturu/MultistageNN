\documentclass[a4paper,12pt]{article}

% Packages
\usepackage{amsmath,amssymb,amsfonts}
\usepackage{graphicx}
\usepackage{hyperref}
\usepackage{xcolor}
\usepackage{geometry}
\usepackage{titlesec}
\usepackage{fancyhdr}
\usepackage{setspace}
\usepackage{enumitem}
\usepackage{caption}

% Page Layout
\geometry{left=1in, right=1in, top=1in, bottom=1in}
\setstretch{1.5}

% Section Formatting
\titleformat{\section}{\large\bfseries}{\thesection.}{0.5em}{}
\titleformat{\subsection}{\normalsize\bfseries}{\thesubsection.}{0.5em}{}
\titleformat{\subsubsection}{\normalsize\itshape}{\thesubsubsection.}{0.5em}{}

% Header & Footer
\pagestyle{fancy}
\fancyhf{}
\fancyhead[L]{\textit{Research Paper Title}}
\fancyhead[R]{\thepage}
\renewcommand{\headrulewidth}{0.4pt}

% Title
\title{\LARGE \textbf{Title of the Research Paper} \\[1ex] \large Subtitle if Needed}
\author{
    First Author$^{1}$ \quad Second Author$^{2}$ \quad Third Author$^{3}$ \\[1ex]
    \small $^{1}$Department, Institution, Email \\
    \small $^{2}$Department, Institution, Email \\
    \small $^{3}$Department, Institution, Email
}
\date{}

\begin{document}

\maketitle

\begin{abstract}
    \noindent This paper presents a study on \textit{(topic of research)}. We explore \textit{(main focus)} and provide insights into \textit{(key contributions)}. Our approach involves \textit{(methods used)}, and we evaluate its performance using \textit{(datasets, experiments, etc.)}. The results indicate \textit{(main findings)}.
\end{abstract}

\section{Introduction}
\noindent The introduction provides an overview of the problem, research motivation, and objectives. It typically includes:

\begin{itemize}
    \item \textbf{Background:} Brief discussion on the topic.
    \item \textbf{Research Problem:} The gap in existing research.
    \item \textbf{Contributions:} What this paper offers.
    \item \textbf{Structure:} Outline of the paper sections.
\end{itemize}

\section{Related Work}
\noindent A review of prior research relevant to this study. This section compares existing methodologies and highlights the need for the proposed approach.

\section{Methodology}
\noindent This section describes the proposed method in detail. It includes:

\subsection{Mathematical Formulation}
Key equations, models, or theoretical background.

\subsection{Proposed Approach}
A detailed description of the method, including algorithms, architecture, or framework.

\section{Experiments and Results}
\noindent This section presents the experimental setup, datasets, performance metrics, and results.

\subsection{Experimental Setup}
Hardware/software specifications, dataset details, and experiment design.

\subsection{Results and Analysis}
Tables, figures, and discussion of findings.

\section{Conclusion and Future Work}
\noindent A summary of key insights and potential future extensions of the work.

\section*{Acknowledgments}
\noindent This section includes funding sources or credits for contributions.

\bibliographystyle{plain}
\bibliography{references}

\end{document}
